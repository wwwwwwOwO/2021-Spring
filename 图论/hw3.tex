\documentclass[UTF8]{ctexart}

%\documentclass[12pt]{article}
\usepackage[a4paper,
            left=0.5in,right=0.5in,top=0.5in,bottom=0.5in]{geometry}

\nonstopmode
%\usepackage[utf-8]{inputenc}
\usepackage{graphicx} % Required for including pictures
\usepackage[figurename=Figure]{caption}
\usepackage{float}    % For tables and other floats
\usepackage{verbatim} % For comments and other
\usepackage{amsmath}  % For math
\usepackage{amssymb}  % For more math
\usepackage{fullpage} % Set margins and place page numbers at bottom center
\usepackage{paralist} % paragraph spacing
\usepackage{listings} % For source code
\usepackage{subfig}   % For subfigures
%\usepackage{physics}  % for simplified dv, and 
\usepackage{enumitem} % useful for itemization
\usepackage{siunitx}  % standardization of si units
\usepackage{graphicx}
\usepackage{hyperref}

\usepackage{tikz,bm} % Useful for drawing plots
%\usepackage{tikz-3dplot}

\begin{document}

\begin{center}
	\hrule
	\vspace{.4cm}
	{\textbf { \Large 图论及其应用:第三次作业}}
\end{center}
	\hrule
	
\vspace{0.5cm}
	
\noindent 请回答任意五题,将答案于北京时间六月三十日午夜前发送至\url{2160853158@qq.com},邮件题目请注明姓名学号 

\vspace{0.5cm}

\paragraph*{题一}
最多可以将地球分成几个区域,使任何两个区域都相邻。

\paragraph*{题二}
证明有10个顶点的5正则图不是平面图。

\paragraph*{题三} 	
考察图$G\triangleq(V,E)$,记$\chi(G)$为$G$的点色数,证明:
\begin{itemize}
    \item 如果$\forall v\in V:\chi(G-v)=\chi(G)-1$,$G$连通;
    \item 如果$\forall x,y\in V:\chi(G-x-y)=\chi(G)-2$,$G$是完全图。
\end{itemize}


\paragraph*{题四} 	
图$G$有$n$个顶点,记$\bar{G}$为$G$的补图,证明:
\begin{itemize}
    \item $\chi(G)\chi(\bar{G})\ge n$;
    \item $\chi(G)+\chi(\bar{G})\le n+1$。
\end{itemize}

\paragraph*{题五} 
3正则图$G$的边色数为$4$,证明$G$不是H图。


\paragraph*{题六} 
给定一个点色数为$k$的$k$染色方案,证明对任何一种颜色$c$,均存在$c$颜色的顶点,其邻居包含所有其他颜色。


\paragraph*{题七} 
给定$n$个顶点,$m$条边的图$G$,证明$G$包含一个偶子图$H$,其边的数目至少为$\displaystyle\frac{2\lfloor n^2/4\rfloor m}{n(n-1)}$。


\paragraph*{题八} 
证明任何平面图最少有$4$个度数小于$6$的顶点。


\paragraph*{题九} 
对$p=1/n$的随机图$G_{n,p}$,证明$\forall\epsilon>0$,大概率不存在多于$(1+\epsilon)n/2$个顶点的连通分支。


\end{document}

